\documentclass[a4paper,12pt]{article}
\usepackage[utf8]{inputenc}
\usepackage{xeCJK}  % 中文支持
\usepackage{amsmath,amsthm,amssymb}  % 数学包
\usepackage{graphicx}  % 插入图片
\usepackage{booktabs}  % 美化表格
\usepackage{hyperref}  % 超链接
\usepackage{enumitem}  % 自定义列表
\usepackage[left=3.5cm,right=3cm,top=2cm,bottom=2cm]{geometry}  % 页面设置
\usepackage{listings}
\usepackage{color}
 
\lstset{    
 frame=none,                                          % 不显示背景边框
 backgroundcolor=\color[RGB]{245,245,244},            % 设定背景颜色
 keywordstyle=\color[RGB]{40,40,255},                 % 设定关键字颜色
 numberstyle=\footnotesize\color{darkgray},           
 commentstyle=\it\color[RGB]{0,96,96},                % 设置代码注释的格式
 stringstyle=\rmfamily\slshape\color[RGB]{128,0,0},   % 设置字符串格式
 showstringspaces=false,                              % 不显示字符串中的空格                                        % 设置语言
}

\begin{document}
 
\section{baidu-serp-api}
\subsection{安装}
\begin{lstlisting}[language=bash]
pip install baidu-serp-api
\end{lstlisting}

\subsection{使用}
\begin{lstlisting}[language=python]
from baidu_serp_api import BaiduPc
import time

pc_serp = BaiduPc()
max_page = 10
max_retry = 5
page = 1
retry = 0

while page <= max_page:
    try:
        results = pc_serp.search('python', 
            pn=page, 
            exclude=['recommend', 'match_count'])
        
        if len(results['data']['results']) == 0:
            if retry < max_retry:
                retry += 1
                time.sleep(5)
                print(f'Retrying... ({retry}/{max_retry})')
                continue
            else:
                print('Max retries reached. Exiting...')
                break
        
        print(f'Page {page}')
        for result in results['data']['results']:
            print(result['title'])

        if results['data']['last_page']:
            break
        page += 1
        retry = 0
        time.sleep(2)
    except Exception as e:
        retry += 1
        print(f'Attempt {retry}/{max_retry} failed: {e}')
        if retry < max_retry:
            time.sleep(5)
        else:
            print('Max retries reached.')
            break
\end{lstlisting}

\subsection{参数}
\begin{itemize}
    \item \verb|keyword|:搜索关键词
    \item \verb|date_range|(可选):搜索日期范围,格式为\verb|'20250501,20250507'|
    \item \verb|pn|(可选):搜索结果页码
    \item \verb|proxies|(可选):代理设置
    \item \verb|exclude|(可选):排除的结果类型
\end{itemize}

\subsection{返回结果}
返回格式为\verb!dict[str, Any] | Any!
\begin{itemize}
    \item \verb|code|:请求状态
    \item \verb|msg|:请求信息
    \item \verb|data|:返回数据
    \begin{itemize}
        \item \verb|results|:搜索结果列表
        \item \verb|recommend|:推荐相关搜索词
        \item \verb|last_page|:是否为最后一页
    \end{itemize}
\end{itemize}

\subsection{注意事项}
\begin{itemize}
    \item \textbf{请求速率}
    
    QPS约为0.1154,即每次请求间隔约为8.67秒.经常性返回空数据导致额外耗时
    \item \textbf{语言设置}

    默认为中文,无法设置其他语言
    \item \textbf{代理设置}

    \verb|pc_serp.search(...proxies={'http': 'http://yourproxy:port'})|
\end{itemize}

\subsection{官方文档}
\url{https://github.com/ohblue/baidu-serp-api}

\section{duckduckgo-search}
\subsection{安装}
\begin{lstlisting}[language=bash]
pip install duckduckgo-search
\end{lstlisting}

\subsection{使用}
\begin{lstlisting}[language=python]
from duckduckgo_search import DDGS
import time

ddgs = DDGS(timeout=20)
max_retry = 5

retry = 0
while retry < max_retry:
    try:
        results = ddgs.text('python', max_results=10)
        for result in results:
            print(result['title'])
            print(result['href'])
            print(result['body'])
        time.sleep(2)
        break
    except Exception as e:
        retry += 1
        print(f"Attempt {retry}/{max_retry} failed: {e}")
        time.sleep(5)
        continue
else:
    print("Max retries reached.")
\end{lstlisting}

\subsection{参数}
\begin{itemize}
    \item \verb|keyword|:搜索关键词
    \item \verb|max_results|(可选):最大返回结果数,若为\verb|None|,则仅返回第一个响应的结果,默认为\verb|None|
    \item \verb|safe_search|(可选):安全搜索设置,默认为\verb|'moderate'|
    \item \verb|region|(可选):地区,默认为\verb|'wt-wt'|
    \item \verb|proxy|(可选):代理设置
\end{itemize}

\subsection{返回结果}
返回格式为\verb|list[dict[str, Any]]|
\begin{itemize}
    \item \verb|title|:搜索结果标题
    \item \verb|href|:搜索结果链接
    \item \verb|body|:搜索结果摘要
\end{itemize}

\subsection{注意事项}
\begin{itemize}
    \item \textbf{请求速率}
    
    QPS约为0.2277,即每次请求间隔约为4.39秒.频繁使用会触发速率限制
    \item \textbf{语言设置}
    \begin{itemize}
        \item \verb|region='wt-wt'|:默认语言
        \item \verb|region='us-en'|:英文
        \item \verb|region='cn-zh'|:中文
    \end{itemize}
    \item \textbf{代理设置}
    
    \verb|ddgs = DDGS(proxy='http://yourproxy:port')|
\end{itemize}

\subsection{官方文档}
\url{https://github.com/deedy5/duckduckgo_search}

\section{googlesearch}
\subsection{安装}
\begin{lstlisting}[language=bash]
pip install googlesearch-python
\end{lstlisting}

\subsection{使用}
\begin{lstlisting}[language=python]
from googlesearch import search
import time

max_retry = 3
retry = 0

while retry < max_retry:
    try:
        results = search("python", 
            num_results=10, 
            advanced=True)
        for result in results:
            print(result.title)
            print(result.url)
            print(result.description)
        break
    except Exception as e:
        retry += 1
        print(f"Attempt {retry}/{max_retry} failed: {e}")
        if retry < max_retry:
            time.sleep(5)
        else:
            print("Max retries reached. Exiting...")
\end{lstlisting}

\subsection{参数}
\begin{itemize}
    \item \verb|keyword|:搜索关键词
    \item \verb|num_results|(可选):最大返回结果数,默认为10
    \item \verb|lang|(可选):语言设置,默认为\verb|"en"|
    \item \verb|region|(可选):地区设置,默认为\verb|None|
    \item \verb|safe|(可选):安全搜索设置,默认为\verb|"active"|
    \item \verb|advanced|(可选):是否获取更多信息,若为\verb|False|则仅返回\verb|url|,默认为\verb|False|
    \item \verb|proxy|(可选):代理设置
    \item \verb|ssl_verify|(可选):是否验证SSL证书,默认为\verb|False|
\end{itemize}

\subsection{返回结果}
返回结果为\verb!Generator[SearchResult | str, Any, None]!
\begin{itemize}
    \item \verb|title|:搜索结果标题
    \item \verb|url|:搜索结果链接
    \item \verb|description|:搜索结果摘要
\end{itemize}

\subsection{注意事项}
\begin{itemize}
    \item \textbf{请求速率}
    
    QPS约为1.0882,即每次请求间隔约为0.92秒.
    \item \textbf{语言设置}
    \begin{itemize}
        \item \verb|lang='en'|:英文
        \item \verb|lang='zh'|:中文
    \end{itemize}
    也可以修改地区设置
    \begin{itemize}
        \item \verb|region='us'|:美国
        \item \verb|region='cn'|:中国
    \end{itemize}
    \item \textbf{代理设置}
    
    \verb|search('python', proxy='http://yourproxy:port')|
\end{itemize}
    
\subsection{官方文档}
\url{https://github.com/Nv7-GitHub/googlesearch}
\end{document}